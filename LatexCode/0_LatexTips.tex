% !TEX TS-program = xelatex
% !TEX encoding = UTF-8 Unicode
\documentclass[AutoFakeBold]{MyFormat}

\usepackage{listings}

\lstset{
 columns=fixed,       
 numbers=left,                                        % 在左侧显示行号
 numberstyle=\tiny\color{gray},                       % 设定行号格式
 frame=none,                                          % 不显示背景边框
 backgroundcolor=\color[RGB]{245,245,244},            % 设定背景颜色
 keywordstyle=\color[RGB]{40,40,255},                 % 设定关键字颜色
 numberstyle=\footnotesize\color{darkgray},           
 commentstyle=\it\color[RGB]{0,96,96},                % 设置代码注释的格式
 stringstyle=\rmfamily\slshape\color[RGB]{128,0,0},   % 设置字符串格式
 showstringspaces=false,                              % 不显示字符串中的空格
 language=c++,                                        % 设置语言
}

\begin{document}
%=====%
%
%封皮页填写内容
%
%=====%

% 标题样式 使用 \title{{}}; 使用时必须保证至少两个外侧括号
%  如: 短标题 \title{{第一行}},  
% 	      长标题 \title{{第一行}{第二行}}
%             超长标题\tiitle{{第一行}{...}{第N行}}

\title{{Latex使用注意事项}}
\entitle{{Directions for The Use of Latex}}
\author{Sillin Ini\\Pinyi Huang}
\maketitle
\thispagestyle{empty}
\newpage

%生成目录
\tableofcontents
\thispagestyle{empty}
\newpage

%文章主体
\mainmatter




% =======正文从第一章开始
\setcounter{chapter}{0}

\chapter{注意事项}
\section{关于\textbackslash newtoks 的使用}
\par 在class文件中,有时需要使用到\textbackslash newtoks来引入tex变量。
而在class中使用到的变量,均需在tex中被定义,且使用
\textbackslash the\textbackslash ...来进行使用。



\end{document}